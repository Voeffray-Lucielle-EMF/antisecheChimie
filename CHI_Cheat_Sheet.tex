\documentclass{article}
\title{Antisèche de Physique}
\author{Anya Voeffray \thanks{thanks to hours of (stu)dying}}
\date{Septembre 2024 - Juillet 2026}

\usepackage{gensymb}
\usepackage[version=4]{mhchem}

\begin{document}

\begin{titlepage}
\maketitle

\begin{equation}
  Mes capacité en physique = \frac{Motivation \cdot Capacités en sciences}{Année depuis le dernier cours de chimie}
\end{equation}

\end{titlepage}


\section{Atomes, éléments chimiques et tableau périodique}

\subsection{L'atome}

\textbf{Définition de l'atome}

L'\textit{atome} est la particule d'un élément qui forme la plus petite quantité susceptible de se combiner.
C'est le constituant fondamental de la matière. 

\subsubsection{Composition de l'atome}

L'atome est constitué de plusieurs particules élémentaires:

\begin{itemize}
  \item Proton [$p^+$] $\rightarrow$ Une particule chargée positivement. Elle se trouve dans le noyau
  \item Electron [$e^-$] \rightarrow Une particule chargée négativement. Elle gravite autour du noyau
  \item Neutron [n | $n^0$] \rightarrow Une particule sans charge éléctrique qui vient annuler la charge positive des protons.
\end{itemize}

\subsubsection{Caractéristiques de l'atome}



\end{document}

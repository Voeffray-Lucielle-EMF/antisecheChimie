% ---------------------------------------------------------------------------------------------------------------
% Titre: PHY_Cheat_Sheet
% Autrice: Lucielle Anya Alita Ahsoka Voeffray
% Date de création: 03.09.25
% Dernière modification: 10.09.25
% Description: Fichier LaTex pour la génération de l'antisèche de physique pour la 3ème année de maturité à l'EMF
% ---------------------------------------------------------------------------------------------------------------

% pdflatex -synctex=1 -interaction=nonstopmode "CHI_Cheat_Sheet".tex

\documentclass{article}
\title{Antisèche de Physique}
\author{Anya Voeffray \thanks{thanks to hours of (stu)dying}}
\date{Septembre 2024 - Juillet 2026}

\usepackage{gensymb}
\usepackage[version=4]{mhchem}

\begin{document}

\begin{titlepage}
\maketitle

\begin{equation}
  Mes capacité en physique = \frac{Motivation \cdot Capacités  en  sciences}{Année  depuis  le  dernier  cours  de  chimie}
\end{equation}

\end{titlepage}


\section{Atomes, éléments chimiques et tableau périodique}

\subsection{L'atome}

\textbf{Définition de l'atome}

L'\textit{atome} est la particule d'un élément qui forme la plus petite quantité susceptible de se combiner.
C'est le constituant fondamental de la matière. 

\subsubsection{Caractéristiques de l'atome}

L'atome est constitué de plusieurs particules élémentaires:

\begin{itemize}
  \item Proton [$p^+$] $\rightarrow$ \text{Une particule chargée positivement. Elle se trouve dans le noyau}
  \item Electron [$e^-$] \rightarrow \text{Une particule chargée négativement. Elle gravite autour du noyau}
  \item Neutron [$n$ | $n^0$] \rightarrow \text{Une particule sans charge éléctrique qui vient annuler celle des $p^+$.}
\end{itemize}

Le noyau de l'atome est formé de nucléons, c'est à dire de neutrons et de protons.
L'enveloppe électronique ou sphère électronique contient les éléctrons
\bigbreak
\textbf{Masse des particules élémentaires}

\begin{table}[-h]
  \begin{center}
    \begin{tabular}{||c|c|c|c||}
      \hline
      Particule & Symbole   & Masse en g      & Masse en u         \\
      électron  & $e^-$     & $9.110e^{-28}$ g & $\frac{1}{1800}$ u \\
      proton    & $p^+$     & $1.673e^{-24}$ g & ~1 u               \\
      neutron   & n / $n^0$ & $1.675e^{-24}$ g & ~1 u              \\
      \hline
    \end{tabular}
  \end{center}
\end{table}

\newline

Donc 1u = $\frac{1}{12}$ de la masse du $\ce{^{12}C}$ et c'est donc égal à $1.6605e^{-24}$ g
\bigbreak
\textbf{Charge des particules élémentaires}

\begin{table}[-h]
  \begin{center}
    \begin{tabular}{||c|c|c|c||}
      \hline
      Particule & Symbole & Charge en C   & Charge en q \\
      électron  & $e^-$   & $-1.6e^{-19}$ & -1q / $q^-$ \\
      proton    & $p^+$   & $1.6e^{-19}$  & 1q / $q^+$  \\
      neutron   & n       & 0C            & 0q         \\
      \hline
    \end{tabular}
  \end{center}
\end{table}

\begin{equation}
  1q = 1.602e^{-19} C
\end{equation}

(1 Coulomb (C) est la quantité d'électricité traversant une section d'un
conducteur parcouru par un courant d'intensité de 1 ampère pendant 1 seconde)
\bigbreak
\textbf{Les forces électrostatiques}
\bigbreak
Les électrons sont attirés par les protons, c'est pour cela qu'ils "gravitent"
autour du noyau.

Les protons ne se repoussent pas entre eux car il y a les neutrons qui annulent
la charge des protons

\end{document}
